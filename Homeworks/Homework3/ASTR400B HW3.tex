\documentclass{article}
\usepackage{graphicx} % Required for inserting images
\usepackage[a4paper, left=1in, right=0.5cm]{geometry}

\title{ASTR4000B HW3}
\author{Laura Johanna Mack}
\date{February 2025}

\begin{document}

\maketitle


\begin{center}
    

\begin{tabular}{||c| c c c |c||c||} 
 \hline
 Galaxy Name & Halo Mass  & Disk Mass  & Bulge Mass  & Total   & Mass fraction\\ [0.5ex] 
  & $(M_\odot$x $10^{12})$ & $(M_\odot$x$10^{12})$ & $(M_\odot$x$10^{12})$ & $(M_\odot$x$10^{12})$\\
 \hline\hline
 Milky Way & 1.975 & 0.075 & 0.01 & 2.06 & 0.041\\ 
 \hline
 M31 & 1.921 & 0.12 & 0.019 & 2.06 & 0.067\\
 \hline
 M33 & 0.187 & 0.009 & 0 & 0.196 & 0.046\\
 \hline\hline
 Total & 408.241 & 20.43 & 2.906 & 431.57 & 0.05\\ 
 \hline 
\end{tabular}
\end{center}

1. The Milky Way and M31 have the same total mass. In both cases, the Halo mass dominates. \\ \\
2. MW has a stellar mass of 0.085 $10^{12}$ x $M_\odot$ and M31 has a stellar mass of 0.139 $10^{12} $x$ M_\odot$. M31 has a larger stellar mass than the Milky Way, so I would expect M31 to be more luminous. \\ \\
3. The baryon mass fraction of the Milky Way and M31 is 0.041 and 0.067, respectively. Of the galaxies investigated, M31 has the highest mass fraction. The fact that it is 2\% larger than the other galaxies is somewhat surprising. Especially given the fact that the Milky Way and M33 have mass fractions that are so similar. \\ \\
4. The universal baryon fraction of 16\% is much larger than the total 5\% present in the simulation. For the Milky Way, with its mass fraction of 4.1\% and M33 with 4.6\% this disparity is greater than for M31 with its larger mass fraction of 6.7\%. To avoid this disparity, there must be some amount of baryons present in the universe outside of galaxies. This is the intergalactic medium. 

\end{document}
